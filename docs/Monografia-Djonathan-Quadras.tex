% Options for packages loaded elsewhere
\PassOptionsToPackage{unicode}{hyperref}
\PassOptionsToPackage{hyphens}{url}
%
\documentclass[
]{book}
\usepackage{amsmath,amssymb}
\usepackage{lmodern}
\usepackage{ifxetex,ifluatex}
\ifnum 0\ifxetex 1\fi\ifluatex 1\fi=0 % if pdftex
  \usepackage[T1]{fontenc}
  \usepackage[utf8]{inputenc}
  \usepackage{textcomp} % provide euro and other symbols
\else % if luatex or xetex
  \usepackage{unicode-math}
  \defaultfontfeatures{Scale=MatchLowercase}
  \defaultfontfeatures[\rmfamily]{Ligatures=TeX,Scale=1}
\fi
% Use upquote if available, for straight quotes in verbatim environments
\IfFileExists{upquote.sty}{\usepackage{upquote}}{}
\IfFileExists{microtype.sty}{% use microtype if available
  \usepackage[]{microtype}
  \UseMicrotypeSet[protrusion]{basicmath} % disable protrusion for tt fonts
}{}
\makeatletter
\@ifundefined{KOMAClassName}{% if non-KOMA class
  \IfFileExists{parskip.sty}{%
    \usepackage{parskip}
  }{% else
    \setlength{\parindent}{0pt}
    \setlength{\parskip}{6pt plus 2pt minus 1pt}}
}{% if KOMA class
  \KOMAoptions{parskip=half}}
\makeatother
\usepackage{xcolor}
\IfFileExists{xurl.sty}{\usepackage{xurl}}{} % add URL line breaks if available
\IfFileExists{bookmark.sty}{\usepackage{bookmark}}{\usepackage{hyperref}}
\hypersetup{
  pdftitle={Otimização Adaptativa Baseada em Simulação para a programação da produção em sistemas flow shop: um estudo comparativo},
  pdfauthor={Djonathan Luiz de Oliveira Quadras},
  hidelinks,
  pdfcreator={LaTeX via pandoc}}
\urlstyle{same} % disable monospaced font for URLs
\usepackage{longtable,booktabs,array}
\usepackage{calc} % for calculating minipage widths
% Correct order of tables after \paragraph or \subparagraph
\usepackage{etoolbox}
\makeatletter
\patchcmd\longtable{\par}{\if@noskipsec\mbox{}\fi\par}{}{}
\makeatother
% Allow footnotes in longtable head/foot
\IfFileExists{footnotehyper.sty}{\usepackage{footnotehyper}}{\usepackage{footnote}}
\makesavenoteenv{longtable}
\usepackage{graphicx}
\makeatletter
\def\maxwidth{\ifdim\Gin@nat@width>\linewidth\linewidth\else\Gin@nat@width\fi}
\def\maxheight{\ifdim\Gin@nat@height>\textheight\textheight\else\Gin@nat@height\fi}
\makeatother
% Scale images if necessary, so that they will not overflow the page
% margins by default, and it is still possible to overwrite the defaults
% using explicit options in \includegraphics[width, height, ...]{}
\setkeys{Gin}{width=\maxwidth,height=\maxheight,keepaspectratio}
% Set default figure placement to htbp
\makeatletter
\def\fps@figure{htbp}
\makeatother
\setlength{\emergencystretch}{3em} % prevent overfull lines
\providecommand{\tightlist}{%
  \setlength{\itemsep}{0pt}\setlength{\parskip}{0pt}}
\setcounter{secnumdepth}{5}
\usepackage{booktabs}
\ifluatex
  \usepackage{selnolig}  % disable illegal ligatures
\fi
\usepackage[]{natbib}
\bibliographystyle{apalike}

\title{Otimização Adaptativa Baseada em Simulação para a programação da produção em sistemas flow shop: um estudo comparativo}
\author{Djonathan Luiz de Oliveira Quadras}
\date{2021-08-30}

\begin{document}
\maketitle

{
\setcounter{tocdepth}{1}
\tableofcontents
}
\hypertarget{apresentauxe7uxe3o}{%
\chapter*{Apresentação}\label{apresentauxe7uxe3o}}
\addcontentsline{toc}{chapter}{Apresentação}

Trabalho Conclusão de Curso de Graduação em Engenharia de Produção Elétrica do Centro Tecnológico da Universidade Federal de Santa Catarina como requisito para a obtenção do título de Bacharel em Engenharia Elétrica com Habilitação em Engenharia de Produção.

\textbf{Orientador:} Prof.~Dr.-Ing Enzo Morosini Frazzon.

\textbf{Coorientador:} Me. Lúcio Galvão Mendes.

Este trabalho é dedicado em memória do meu bocó, Valerio Quadras.

\hypertarget{agradecimentos}{%
\section*{Agradecimentos}\label{agradecimentos}}
\addcontentsline{toc}{section}{Agradecimentos}

Agradeço aos meus pais, Luiz e Eliana, e aos meus avós, Valerio (in memoriam) e Terezinha, por todo o apoio, por estarem sempre comigo me apoiando, e por sempre terem acreditado em mim. Agradeço à minha namorada, Camila Bertelli, por toda a ajuda (inclusive na elaboração deste trabalho), todo o companheirismo, toda a compreensão e, principalmente, por ter me feito feliz até nos períodos em que não achei que conseguiria. Agradeço à minha sogra, Roseli, e ao meu cunhado, Matheus, por me acolherem e serem minha segunda família.

Agradeço a todos os meus amigos que Florianópolis me proporcionou, e principalmente aos meus companheiros da República da Chatuba, André, Gabriel Diz e Renan, e ao meu primo que tive o prazer de reencontrar, Valerio. Vocês fizeram todos os dias valerem a pena, e me mostraram que mesmo a pior das situações devem ser vistas como uma piada em potencial. Agradeço a todos os companheiros que estiveram comigo em minha passagem pela Colômbia, em especial a minha querida amiga Giovanna, minha fiel escudeira, o Sancho da minha Pança.

Agradeço a todos os meus companheiros do ProLogIS, em especial à Marina, ao Ícaro, ao Matheus e ao Lúcio. Sem vocês eu não conseguiria estar aqui hoje.

Agradeço a todos os professores da UFSC, CUC e UDESC que fizeram parte de todo este trajeto. Em especial ao professor Enzo por todos estes anos de trabalho em conjunto, por toda a confiança e por todos os ensinamentos; e ao professor Carlos Taboada por toda a amizade e por ter me proporcionado minha primeira experiência internacional (e uma das melhores experiências de minha vida). Definitivamente, ``se cheguei até aqui, foi porque me apoiei sobre o ombro de gigantes''.

Por fim, agradeço a todos aqueles que defendem e acreditam na ciência e que combatem diariamente o obscurantismo científico.

\begin{quote}
``Pra entender, basta seiscentos anos de estudo\ldots{} Ou seis segundos de atenção.''

\hfill --- GESSINGER, 1988
\end{quote}

\hypertarget{resumo}{%
\chapter*{Resumo}\label{resumo}}
\addcontentsline{toc}{chapter}{Resumo}

Working on it :)

\hypertarget{introduuxe7uxe3o}{%
\chapter{Introdução}\label{introduuxe7uxe3o}}

Working on it :)

\hypertarget{revisuxe3o-da-literatura}{%
\chapter{Revisão da Literatura}\label{revisuxe3o-da-literatura}}

Working on it :)

\hypertarget{metodologia}{%
\chapter{Metodologia}\label{metodologia}}

Working on it :)

\hypertarget{resultados-e-discussuxf5es}{%
\chapter{Resultados e Discussões}\label{resultados-e-discussuxf5es}}

Working on it :)

\hypertarget{conclusuxe3o}{%
\chapter{Conclusão}\label{conclusuxe3o}}

Working on it :)

\hypertarget{referuxeancias-bibliogruxe1ficas}{%
\chapter*{Referências Bibliográficas}\label{referuxeancias-bibliogruxe1ficas}}
\addcontentsline{toc}{chapter}{Referências Bibliográficas}

Working on it :)

  \bibliography{book.bib}

\end{document}
