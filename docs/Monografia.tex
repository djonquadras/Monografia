% Options for packages loaded elsewhere
\PassOptionsToPackage{unicode}{hyperref}
\PassOptionsToPackage{hyphens}{url}
%
\documentclass[
]{book}
\usepackage{amsmath,amssymb}
\usepackage{lmodern}
\usepackage{ifxetex,ifluatex}
\ifnum 0\ifxetex 1\fi\ifluatex 1\fi=0 % if pdftex
  \usepackage[T1]{fontenc}
  \usepackage[utf8]{inputenc}
  \usepackage{textcomp} % provide euro and other symbols
\else % if luatex or xetex
  \usepackage{unicode-math}
  \defaultfontfeatures{Scale=MatchLowercase}
  \defaultfontfeatures[\rmfamily]{Ligatures=TeX,Scale=1}
\fi
% Use upquote if available, for straight quotes in verbatim environments
\IfFileExists{upquote.sty}{\usepackage{upquote}}{}
\IfFileExists{microtype.sty}{% use microtype if available
  \usepackage[]{microtype}
  \UseMicrotypeSet[protrusion]{basicmath} % disable protrusion for tt fonts
}{}
\makeatletter
\@ifundefined{KOMAClassName}{% if non-KOMA class
  \IfFileExists{parskip.sty}{%
    \usepackage{parskip}
  }{% else
    \setlength{\parindent}{0pt}
    \setlength{\parskip}{6pt plus 2pt minus 1pt}}
}{% if KOMA class
  \KOMAoptions{parskip=half}}
\makeatother
\usepackage{xcolor}
\IfFileExists{xurl.sty}{\usepackage{xurl}}{} % add URL line breaks if available
\IfFileExists{bookmark.sty}{\usepackage{bookmark}}{\usepackage{hyperref}}
\hypersetup{
  pdftitle={Aplicações de Ferramentas Computacionais e Estatísticas em um Escritório de Advocacia},
  pdfauthor={Djonathan Luiz de Oliveira Quadras},
  hidelinks,
  pdfcreator={LaTeX via pandoc}}
\urlstyle{same} % disable monospaced font for URLs
\usepackage{longtable,booktabs,array}
\usepackage{calc} % for calculating minipage widths
% Correct order of tables after \paragraph or \subparagraph
\usepackage{etoolbox}
\makeatletter
\patchcmd\longtable{\par}{\if@noskipsec\mbox{}\fi\par}{}{}
\makeatother
% Allow footnotes in longtable head/foot
\IfFileExists{footnotehyper.sty}{\usepackage{footnotehyper}}{\usepackage{footnote}}
\makesavenoteenv{longtable}
\usepackage{graphicx}
\makeatletter
\def\maxwidth{\ifdim\Gin@nat@width>\linewidth\linewidth\else\Gin@nat@width\fi}
\def\maxheight{\ifdim\Gin@nat@height>\textheight\textheight\else\Gin@nat@height\fi}
\makeatother
% Scale images if necessary, so that they will not overflow the page
% margins by default, and it is still possible to overwrite the defaults
% using explicit options in \includegraphics[width, height, ...]{}
\setkeys{Gin}{width=\maxwidth,height=\maxheight,keepaspectratio}
% Set default figure placement to htbp
\makeatletter
\def\fps@figure{htbp}
\makeatother
\setlength{\emergencystretch}{3em} % prevent overfull lines
\providecommand{\tightlist}{%
  \setlength{\itemsep}{0pt}\setlength{\parskip}{0pt}}
\setcounter{secnumdepth}{5}
\usepackage{booktabs}
\ifluatex
  \usepackage{selnolig}  % disable illegal ligatures
\fi
\usepackage[]{natbib}
\bibliographystyle{apalike}

\title{Aplicações de Ferramentas Computacionais e Estatísticas em um Escritório de Advocacia}
\author{Djonathan Luiz de Oliveira Quadras}
\date{2021-04-11}

\begin{document}
\maketitle

{
\setcounter{tocdepth}{1}
\tableofcontents
}
\hypertarget{bem-vindo}{%
\chapter*{Bem vindo!}\label{bem-vindo}}
\addcontentsline{toc}{chapter}{Bem vindo!}

Este é o site da minha monografia a ser apresentada ao Departamento de Engenharia de Produção e Sistemas (DEPS) da Universidade Federal de Santa Catarina (UFSC). Sinta-se a vontade para me enviar suas sugestões e críticas! :)

Contato: \href{mailto:djonquadras@gmail.com}{\nolinkurl{djonquadras@gmail.com}}

\hypertarget{resumo}{%
\chapter*{Resumo}\label{resumo}}
\addcontentsline{toc}{chapter}{Resumo}

Working on it :)

\hypertarget{introduuxe7uxe3o}{%
\chapter{Introdução}\label{introduuxe7uxe3o}}

A idéia de formalizar os processos de tomada de decisão para que possam ser automatizados por algoritmos tem sido uma idéia atraente para muitos estudiosos e profissionais da área jurídica \citep{Waltl2018, liu2004}. De fato, em 1991 Fiedler já defendia a utilização de sistemas inteligentes e a aplicação de técnicas de inteligência artificial (IA) nos sistemas jurídiculos em detrimento da utilização de sistemas puramente ``organizadores''.

\citet{Fiedler1991} defende que a utilização da lógica no direito é algo necessário e aproxima a ideia de lógica jurídica com a lógica matemática.

\citet{Guarino2019} defende que o desenvolvimento de ferramentas inovadoras para a análise e processamento de dados é um dos principais desafios do desenvolvimento de Tecnologias da Informação e Comunicação (TIC).

\citet{Colombo2017} afirma que o interesse de desenvolver pesquisas quantitativas na área jurídica surgiu no Brasil apenas em 2011, quando os primeiros artigos foram publicados em periódicos locais.

\hypertarget{revisuxe3o-da-literatura}{%
\chapter{Revisão da Literatura}\label{revisuxe3o-da-literatura}}

\hypertarget{citauxe7uxf5es-gerais}{%
\section{Citações Gerais}\label{citauxe7uxf5es-gerais}}

\citet{Saarenpaä2018} afirma que sistemas de informação devem sempre ser planejados para minizar o risco de erros humanos e, consequentemente, minimar os impactos dos erros.

\citet{Saarenpaä2018} afirma que há muitos tipos de advogados hoje, incluindo aqueles que aderem às novidades tecnológicas e aqueles que as rejeitam. O autor cita que rejeitar as novidades tecnológicas é um tremendo erro uma vez que ``\emph{nada é criado no vácuo}'' e tudo necessita de uma influência.

\citet{Saarenpaä2018} defende que a lei e os advogados trabalham combinando premissas factuais e normativas.

\citet{Waltl2018} defendem sobre a Tomada de Decisão Algorítmica (TDA) e sugere que existe duas razões para sua aplicação: (1) a habilidade de entendimento fácil e consequente representação do conhecimento e (2) a explicabilidade e transparência de decisões.

Na mesma linha de raciocínio da TDA, \citet{liu2004} sugere a utilização do Raciocínio Baseado em Casos (RBC). Eles aplicaram tecologias de inteligência artificial para classificar dados e informações de julgamentos criminais em Taiwan.

\hypertarget{jurimetria-e-informuxe1tica-juruxeddica}{%
\section{Jurimetria e Informática Jurídica}\label{jurimetria-e-informuxe1tica-juruxeddica}}

Jurimetria como a aplicação de métodos quantitativos, geralmente a estatística, no direito \citep{Colombo2017}.

\citet{Colombo2017} afirma que utilizar métodos quantitativos para analizar decições jurídicas torna possível identificar padrões e \emph{outliers}, sendo, assim, possível prever a decisão de um processo.

\citet{Colombo2017} ressaltam um problema importante: os documentos e as informações nas diferentes cortes não são padronizados.

\citet{Saarenpaä2018} afirma que a informática jurídica é mais do que uma simples especialização.

\citet{Waltl2018} dividem nas seguintes abordagens diferentes de raciocínio IA:

\begin{itemize}
\tightlist
\item
  \textbf{Raciocínio Dedutivo}: sistemas de especialistas jurídicos, programação lógica clássica;
\item
  \textbf{Raciocínio Baseado em Casos}: indução de regras com base em casos anteriores e precedentes;
\item
  \textbf{Raciocínio Abdutivo}: vinculação semântica, encontrar explicações simples e prováveis;
\item
  \textbf{Raciocínio Viável}: lógicas não monotônicas e argumentação;
\item
  \textbf{Raciocínio Probabilístico}: lógica difusa, raciocínio em termos indeterminados e vagos;
\item
  \textbf{Raciocínio Sobre Ontologias}: representações formais do conhecimento, rede semântica;
\item
  \textbf{Raciocínio Estatístico}: incluindo abordagens de aprendizado de máquina (ML não supervisionado);
\item
  \textbf{Aprendizado Avançado de Máquina}: aprendizado ativo, interativo e de reforço.
\end{itemize}

\hypertarget{artigos-que-apresentam-softwares}{%
\subsection{\texorpdfstring{Artigos que apresentam \emph{softwares}}{Artigos que apresentam softwares}}\label{artigos-que-apresentam-softwares}}

\citet{Guarino2019} apresentam o programa \emph{Argos}, uma plataforma de análise visual para dados abministrativos.

\hypertarget{proteuxe7uxe3o-de-dados}{%
\subsection{Proteção de dados}\label{proteuxe7uxe3o-de-dados}}

Revisão

\hypertarget{a-ciuxeancia-de-dados}{%
\section{A Ciência de Dados}\label{a-ciuxeancia-de-dados}}

\citet{Waltl2018} alertam que os sistemas inteligentes e que regularmente são utilizados pela ciência de dados parecem muito com ``caixas pretas'' onde ocorre uma ``mágica'' que devolve um resultado surpreendente. Isso gera um desinteresse em compreender profundamente o que acontece por trás de toda a programação.

\hypertarget{modelos-estatuxedsticos-e-de-inteliguxeancia-artificial}{%
\subsection{Modelos Estatísticos e de Inteligência Artificial}\label{modelos-estatuxedsticos-e-de-inteliguxeancia-artificial}}

Explicação dos modelos

\hypertarget{metodologia}{%
\chapter{Metodologia}\label{metodologia}}

\citet{Waltl2018} Concorda com Wickham quato à metodolodia aplicada à ciência de dados, explicitando que as etapas são:

\begin{itemize}
\tightlist
\item
  Aquisição dos Dados;
\item
  Pré-processamento dos dados;
\item
  Transformação dos Dados;
\item
  Treinamento e aplicação do modelo de Inteligência Artificial;
\item
  Interpretação e avaliação.
\end{itemize}

Working on it :)

\hypertarget{resultados}{%
\chapter{Resultados}\label{resultados}}

Working on it :)

\hypertarget{discussuxf5es-e-conclusuxe3o}{%
\chapter{Discussões e Conclusão}\label{discussuxf5es-e-conclusuxe3o}}

Working on it :)

  \bibliography{book.bib}

\end{document}
